\documentclass[11pt]{article}
\usepackage[utf8]{inputenc}
\usepackage[T1]{fontenc}

% Page Layout
\usepackage[top=2cm, bottom=2cm, left=2.5cm, right=5cm]{geometry} % Right margin for margin notes
\usepackage{fancyhdr}
\pagestyle{fancy}
\fancyhf{}
\renewcommand{\headrulewidth}{0.4pt}
\fancyhead[L]{\leftmark} % Current section
\fancyhead[R]{\rightmark} % Current subsection
\fancyfoot[C]{\thepage}

% Math Packages
\usepackage{amsmath, amssymb, amsthm}
\usepackage{mathtools}

% Theorems & Definitions
\theoremstyle{definition}
\newtheorem{definition}{Definition}[section]
\newtheorem{example}{Example}[section]

\theoremstyle{plain}
\newtheorem{theorem}{Theorem}[section]
\newtheorem{corollary}{Corollary}[theorem]
\newtheorem{lemma}[theorem]{Lemma}

% Visual Elements
\usepackage{xcolor}
\usepackage{marginnote}
\usepackage{tcolorbox}
\tcbuselibrary{breakable}

% Custom Commands
\newcommand{\important}[1]{\textcolor{red}{\textbf{#1}}}
\newcommand{\todo}[1]{\marginpar{\textcolor{red}{TODO: #1}}}
\newcommand{\keyterm}[1]{\textbf{\textcolor{blue}{#1}}}

% Hyperlinks
\usepackage{hyperref}
\hypersetup{
    colorlinks=true,
    linkcolor=blue,
    filecolor=magenta,      
    urlcolor=cyan,
}

\begin{document}

% Title Page
\title{Class Notes: CSCI - 395}
\author{Your Name}
\date{\today}
\maketitle
\tableofcontents
\newpage

% Sample Lecture
\section{Lecture 1: Introduction (Date)}
\subsection{Main Topics}

\begin{definition}[Important Concept]
This is where you define key concepts. \keyterm{Keyword} highlighted.
\end{definition}

\begin{theorem}[Pythagorean Theorem]
For a right triangle:
\begin{equation}
a^2 + b^2 = c^2
\end{equation}
\end{theorem}

\marginnote{-15mm}{\textcolor{blue}{Remember!} This is a margin note for quick reminders.}

\begin{example}
Calculate hypotenuse when $a=3$, $b=4$:
\begin{equation*}
c = \sqrt{3^2 + 4^2} = 5
\end{equation*}
\end{example}

\subsection{Summary}
\begin{itemize}
\item First key point \todo{Add reference}
\item Second important idea
\item \important{Crucial takeaway}
\end{itemize}

% Additional Features
\subsection{Extra Tools}
\begin{tcolorbox}[title=Quick Reference,colback=blue!5!white,colframe=blue!75!black]
This is a colored box for important information.
\end{tcolorbox}

\end{document}